\input regression-test

\title{Moloch Test Suite}
\subtitle{A subtitle that is way to long and in fact might just need to be split across lines}

\author[Johan]{Johan Larsson}
\institute[LU]{Lund Univesity//Department of Statistics}
\date{April 23, 2024}

\begin{document}

\START
\showoutput

\begin{frame}{Table of contents}
  \setbeamertemplate{section in toc}[sections numbered]
  \tableofcontents[hideallsubsections]
\end{frame}

\section{Results}

\subsection{Proof of the Main Theorem}

\begin{frame}<1>
  \frametitle{There Is No Largest Prime Number}
  \framesubtitle{The proof uses \textit{reductio ad absurdum}.}

  \begin{theorem}
    There is no largest prime number.
  \end{theorem}
  \begin{proof}
    \begin{enumerate}
      \item<1-| alert@1> Suppose $p$ were the largest prime number.
      \item<2-> Let $q$ be the product of the first $p$ numbers.
      \item<3-> Then $q$\;+\,$1$ is not divisible by any of them.
      \item<1-> Thus $q$\;+\,$1$ is also prime and greater than $p$.\qedhere
    \end{enumerate}
  \end{proof}
\end{frame}

\vfil\break
\END

\begin{frame}{Lists}
  \begin{columns}[T,onlytextwidth]
    \column{0.33\textwidth}
    Items
    \begin{itemize}
      \item Milk \item Eggs \item Potatoes
    \end{itemize}

    \column{0.33\textwidth}
    Enumerations
    \begin{enumerate}
      \item First, \item Second and \item Last.
    \end{enumerate}

    \column{0.33\textwidth}
    Descriptions
    \begin{description}
      \item[PowerPoint] Meeh. \item[Beamer] Yeeeha.
    \end{description}
  \end{columns}
\end{frame}

\begin{frame}{Animation}
  \begin{itemize}[<+- | alert@+>]
    \item \alert<4>{This is\only<4>{ really} important}
    \item Now this
    \item And now this
  \end{itemize}
\end{frame}

\appendix

\begin{frame}{Backup slides}
  Here are some backup slides
\end{frame}

\end{document}

\end{document}
